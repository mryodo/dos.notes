\documentclass{mynotes}


\title{Density of States of Hodge Laplacians: Decomposition effects and Sparsification of SC}

\author[1]{ Tony Savostianov }

\affil[1]{ RWTH, Aachen   \\ email: \email{a.s.savostyanov@gmail.com} }

\keywords{density of states, simplicial complexes, sparsification, generalized effective resistance}

\input{shortcuts.tex}

\begin{document}

\maketitle


\chapter{Introduction}

Well, it would be nice to have one. 

You know, somewhere here, maybe\dots


\chapter{ Simplicial complexes }




\chapter{Sparsification of Simplicial Complex}

Let \( \mc K \) be a simplicial complex with the unit weights of \( \V 1 \), \( W_1 = I \). Then \( \Lu 1 = B_2 W_2^2 B_2^\top \) and the \emph{generalized effective resistance} is given by 
\begin{equation}
      \b r = \diag \left( B_2^\top \left( \Lu 1 \right)^+ B_2 \right) = \diag \left( B_2^\top \left( B_2 W_2^2 B_2^\top \right)^+ B_2 \right)
\end{equation}
then the sparsifying measure is defined as \( \b p \sim \diag \left( W_2^2 \b r  \right)\) (let us temporary believe that this is correct and we do not need to touch it).

\begin{remark}[on the weird-weird-weird matrix inside \( \b r \)]
      Let us take a further look at GER above. Let SVD 
      \( B_2 W_2 = U S V^\top\); then 
      \begin{equation}
            \left( B_2 W_2^2 B_2^\top \right)^+ = U S^{+2} U^\top 
      \end{equation}
      and \( B_2 = U S V^\top W_2^{-1} \). As a result, 
      \begin{equation}
            B_2^\top \left( B_2 W_2^2 B_2^\top \right)^+ B_2 = W_2^{-\top} V S S^{+2} S V^\top W_2^{-1} 
      \end{equation}
      Since \( S \) and \( S^+ \) are diagonal, any permutations of \( S S^{+2} S \) are allowed. Then \( S S^{+2} S = S S^+ \). As a result for GER:  
      \begin{equation}
            \begin{aligned}
                  \b r & = \diag( X X^\top ) \\
                  \text{where } X & = W_2^{-1} V S S^+ = W_2^{-1} V \Pi = W_2^{-1} V_1 
            \end{aligned}
      \end{equation}
      where \( V_1 \) is the orthonormal basis of \( \im B_2^\top \).
\end{remark}





\clearpage
%% BIBLIOGRAPHY %% 
\nocite{*}
\bibliographystyle{alpha}
\bibliography{notes}



\end{document}