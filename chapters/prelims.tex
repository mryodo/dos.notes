\chapter{ Preliminaries }

\section{ Simplicial Complexes, \cite{Lim15} }
%
%    + complex, boundary operators, Laplacians, 
%    + weighted case, _Hodge decomposition_ 
%    + _???_ spectral inheritance _???_
%

\paragraph{Simplicial complexes}

Let \( V = \{ v_1, \dots v_{m_0} \} \) be the set of nodes. Each subset of nodes \( \sigma = \{ v_{i_1}, \dots v_{i_k} \} \) is known as \emph{simplex} of order \( k - 1 \) with all its maximal proper subsets knowns as \emph{faces} of \( \sigma \). Then a collection of simplices \( \mc K \) is a \emph{simplicial complex} if and only if each simplex \( \sigma \) enters \( \mc K \) with all its faces (inclusion rule). As a result, each simplicial complex can be written as a collections of simplices of fixed orders, \( \mc K = \left\{ \V 0, \V 1, \V 2, \dots \right\} \) where \( \V 0 \) is a set of \(0\)-simplices (nodes), \( \V 1 \) is a set of edges, \( \V 2 \) of triangles, and so on, with \( m_k = | \V k | \) denoting the number of simplices in the corresponding set.

\paragraph{ Boundary operators and Laplacians }

Due to the inclusion principle, simplices inside \( \mc K \) form a nested structure through boundary relation; specifically, one can map a \( k \)-order simplex to a collection of \( (k-1)\)-order simplices on its border. Formally, let \( \mc C_k  \) be a linear space of all formal sums of simplices in \( \V k \); then the boundary map \( B_k \) is defined as an alternating sum:
\begin{equation}
      B_k : \mc C_k \mapsto \mc C_{k-1}, \qquad B_k [ v_1, v_2, \ldots v_k ] = \sum_{i=1}^k (-1)^{i-1} [ v_1, \ldots v_{i-1}, v_{i+1}, \ldots v_k ] 
\end{equation}
such that boundary of a boundary is \( 0 \), \( B_k B_{k+1} = 0 \), (fundamental lemma of homology). Provided one fixes an order of simplices and their orientation (lexicographically, so the simplicity's sake), one can define the following entities describing the higher-order structure of \( \mc K \):
\begin{definition}[Homology group and higher-order Laplacian]
      Since \( \im B_{k+1} \subset \ker B_k \), the quotient space \( \mc H_k =  \sfrac{ \ker B_k }{ \im B_{k+1}} \), known as \( k\)-th homology group, is correctly defined and the following isomorphisms hold 
\begin{equation*}
            \mc H_k \cong \ker B_k \cap \ker B_{k+1}^\top \cong \ker \left( B_k^\top B_k + B_{k+1} B_{k+1}^\top \right).
      \end{equation*}

      The matrix \( L_k = B_k^\top B_k + B_{k+1} B_{k+1}^\top \) is called the \(k\)-th order \emph{Laplacian operator}; the two terms \( \Ld k =  B_k^\top B_k \) and \( \Lu k = B_{k+1} B_{k+1}^\top \) are referred to as the \emph{down-Laplacian} and the \emph{up-Laplacian}, respectively.
\end{definition}
The homology group \( \mc H_k \) describes the \(k\)-th topology of the simplicial complex \( \mc K \): \( \beta_k = \dim \mc H_k = \dim \ker L_k \)  coincides exactly with the number of \(k\)-dimensional holes in the complex, known as the \emph{ \(k\)-th Betti number}. In the case \( k = 0 \), the operator \( L_0 = \Lu 0\) is exactly the classical graph Laplacian whose kernel corresponds to the \textit{connected components} of the graph, while \( \Ld 0  = 0 \). For \( k = 1 \) and \( k = 2\), the elements of \( \ker L_1 \) and \( \ker L_2\) describe the simplex 1-dimensional holes and voids respectively, and are frequently used in the analysis of trajectory data,~\cite{schaub2019random,benson2016higher}. \todo{redo}

\paragraph{ Weighted case and spectral structures }

Considering weighted generalization of simplicial complexes, one needs to preserve the fundamental lemma of homology, \( B_k B_{k+1} = 0 \). As a result, let us assume the collection of weight matrices \( W_k \in \ds R_{m_k \times m_k} \) such that  \( W_k \) is diagonal with each \( (W_k)_{ii} > 0 \) containing the weight of the \(i\)-the simplex in \( \V k \). Then a generic weighting scheme for boundary operators can be written as:
\begin{equation}
      B_k \mapsto W_{k-1}^{-1} B_k W_k
\end{equation}
Note that with such weighting scheme the dimensionality of the homology group is preserved, \( \dim \ker L_k = \dim \ker \widehat L_k \),~\cite{guglielmi2023quantifying}.

At each order \( k \), weighted boundary operators \( B_k \) define the space decomposition (Hodge decomposition):
\begin{equation}
      \ds R^{m_k} = \lefteqn{\overbrace{\phantom{\im B_k^\top \oplus  \ker \left( B_k^\top B_k + B_{k+1} B_{k+1}^\top \right)}}^{\ker B_{k+1}^\top}} \im B_k^\top \oplus
      \underbrace{\ker \left( B_k^\top B_k + B_{k+1} B_{k+1}^\top \right) \oplus  \im B_{k+1}}_{\ker B_k}            
\end{equation}
implying that that kernels of up- and down-Laplacians are necessarily non-trivial and high-dimensional, \( \ker \Ld k \supseteq \im B_{k+1} \) and \( \ker \Lu k \subseteq \im B_k^\top \). More explicit relation between these subspaces is known as a principle spectral inheritance, \cite{guglielmi2023quantifying}.

% Let \( V = \{ v_1, \dots v_{m_0} \} \) be the set of nodes in the system; each its subset \( \sigma = \{ v_{i_1}, \dots v_{i_k} \} \subseteq V \) of cardinality \( k \) is referred to as a \emph{simplex} of order \( k - 1 \). The collection \( \mc k \) of simplices on the node set \( V \) is known as a \emph{simplicial complex} if and only if each simplex \( \sigma \) enters it with all its faces (for each \( \sigma' \subseteq \sigma \), \( \sigma' \in \mc K \)). We denote the set of simplices of order \( k \) in \( \mc K \) by \( \V k \), \( \mid \V k | = m_k \); then, any simplicial complex can be thought of as a collection of sets of simplices of \( 0\)-order (nodes), \( 1\)-order (edges), \( 2\)-order (triangles), and so on, \( \mc K = \left\{ \V 0, \V 1, \V 2 \dots \right\}\).

%The face inclusion principle implies the nested structure on \( \V k \); due to it, one can define boundary operators \( B_k : \V k \mapsto \V {k-1} \) mapping simplices to the formal sum of simplices on its boundary. Formally,    

%\begin{figure}[hbtp]
      \centering
      \begin{tikzpicture}
            \begin{scope}[shift={(-0.75, 0)}]
            \draw[fill = liberty, opacity = 0.4] (0,0) -- (1.5,0) -- (0.75, -1.5*3/4) -- cycle; 
            \draw[fill = liberty, opacity = 0.6] (0,0) -- (1.5,0) -- (0.75, 1.5*3/4) -- cycle; 

            \Vertex[x=0, y=0, label = 1, style={color = persimmon}, fontcolor = white, size = 0.4 ]{v1}
            \Vertex[x=1.5, y=0, label = 3, style={color = persimmon}, fontcolor = white, size = 0.4 ]{v2}
            \Vertex[x=0.75, y=-1.5*3/4, label = 2, style={color = persimmon}, fontcolor = white, size = 0.4 ]{v3}
            \Vertex[x=0.75, y=1.5*3/4, label = 4, style={color = persimmon}, fontcolor = white, size = 0.4 ]{v4}
            \Edge[Direct](v1)(v2)
            \Edge[Direct](v1)(v3)
            \Edge[Direct](v1)(v4)
            \Edge[Direct](v3)(v2)
            \Edge[Direct](v2)(v4)
            \node at (0.75, 1.5*3/4*1/3 ) { \AxisRotator[rotate=0] };
            \node at (0.75, -1.5*3/4*1/3 ) { \AxisRotator[rotate=-60] };
            \end{scope}

            \Vertex[x=2.5, y=1.5*3/4, label = 1, style={color = persimmon}, fontcolor = white, size = 0.4 ]{t1}
            \Vertex[x=4, y=1.5*3/4, label = 2, style={color = persimmon}, fontcolor = white, size = 0.4 ]{t2}
            \Vertex[x=2.5, y=0.9*3/4, label = 1, style={color = persimmon}, fontcolor = white, size = 0.4 ]{t3}
            \Vertex[x=4, y=0.9*3/4, label = 3, style={color = persimmon}, fontcolor = white, size = 0.4 ]{t4}
            \Vertex[x=2.5, y=0.3*3/4, label = 1, style={color = persimmon}, fontcolor = white, size = 0.4 ]{t5}
            \Vertex[x=4, y=0.3*3/4, label = 4, style={color = persimmon}, fontcolor = white, size = 0.4 ]{t6}
            \Vertex[x=2.5, y=-0.3*3/4, label = 2, style={color = persimmon}, fontcolor = white, size = 0.4 ]{t7}
            \Vertex[x=4, y=-0.3*3/4, label = 3, style={color = persimmon}, fontcolor = white, size = 0.4 ]{t8}
            \Vertex[x=2.5, y=-0.9*3/4, label = 3, style={color = persimmon}, fontcolor = white, size = 0.4 ]{t9}
            \Vertex[x=4, y=-0.9*3/4, label = 4, style={color = persimmon}, fontcolor = white, size = 0.4 ]{t10}
            \Edge[Direct](t1)(t2)
            \Edge[Direct](t3)(t4)
            \Edge[Direct](t5)(t6)
            \Edge[Direct](t7)(t8)
            \Edge[Direct](t9)(t10)

            \draw[->, line width = 1.0] (2.1, 1.9*3/4)--(2.1, -1.5*3/4);
            \draw[->, line width = 1.0] (2.1, 1.9*3/4)--(4.4, 1.9*3/4);
            \node[ anchor=south ] at ( 3.25, 1.9*3/4 ) { \small orientation };
            \node[ anchor = south, rotate = 90 ] at (2.1, 0.2*3/4 ) { \small ordering };

            \node[] at ( 10.25, 1.0 ) { \( B_2 \textcolor{liberty}{[1, 2, 3 ]} = \overbrace{\textcolor{red}{(+1)} [1, 2]}^{\substack{\text{1st in}\\\text{order}}} + (-1) [1, 3] + (+1) [2, 3] \) };
            \node[] at ( 10.25, -0.4 ) { \( B_2 \textcolor{liberty}{[1, 3, 4 ]} = \underbrace{\textcolor{red}{(+1)} [1, 3]}_{\substack{\text{1st in}\\\text{order}}} + (-1) [1, 4] + (+1) [3, 4] \) };  
      \end{tikzpicture}
      \caption{ Example of the simplicial complex with ordering and orientation: nodes from \( \V 0 \) in orange, triangles from \( \V 2 \) in blue. Orientation of edges and triangles is shown by arrows; the action of \( B_2 \) operator is given for both triangles.\label{fig:orientation}}
\end{figure}



\section{ Networks' Density of States, \cite{dong2019network} }
%
%    1. General Density of States
%    2. Local Density of States
%    3. Histogram

\paragraph{ Spectral densities }

As we established above, the spectral structure of the higher-order Laplacian operators carry meaningful information of the complex's topological structure, YADA YADA \todo{ the usual song}

At the same time, the overall cost of extracting full spectral information from corresponding Laplacian operators \( L_k \) is computationally expensive with a complexity \( \mc O ( m_k^3 ) \). Instead, one can consider the following functional description:

\begin{definition}[Density of States]
      Fro a given symmetric matrix \( A = Q \Lambda Q^\top \) with \( Q^\top Q = I \) and diagonal \( \Lambda = \diag \left( \lambda_1, \dots \lambda_{n} \right) \), the \emph{spectral density} or \emph{density of states} (DoS)
      \begin{equation}
            \mu( \lambda \mid A ) = \frac{1}{n} \sum_{i=1}^{n} \delta \left( \lambda - \lambda_i \right)
      \end{equation}
      Additionally, let \( \b q_i \) be a corresponding unit eigenvector of \( A \) (such that \( A \b q_i = \lambda_i \b q_i \) and \( Q = \left( \b q_1 \mid \b q_2 \mid \dots \mid \b q_n \right)\)); then one can define a set of local (entry-wise) densities of states (\emph{LDoS}):
      \begin{equation}
            \mu_k ( \lambda \mid A ) = \sum_{i=1}^{n} \left| \b e_k^\top \b q_i \right|^2 \delta \left( \lambda - \lambda_i \right)
      \end{equation}
      with \( \b e_k \) being the corresponding versor. 
\end{definition}
Here DoS function \( \mu( \lambda \mid A ) \) contains the overall spectrum of the operator \( A \) while the family of LDoS \( \mu_k ( \lambda \mid A )\) describe the contribution of the simplex \( \sigma_k \in \V k \) to the spectral information.

Finally, one should note that by its definition DoS and LDoS  are generalized functions, hence the quality of their computation is difficult to assess directly; instead, one considers its histogram representations:
\begin{equation}
    h_i = \int_{x_i}^{x_i+\Delta_x} \mu( \lambda \mid A ) d\lambda, \qquad h^{(k)}_i = \int_{x_i}^{x_i+\Delta_x} \mu_k( \lambda \mid A ) d\lambda 
\end{equation}
which correspond to discretized output of the convolution of spectral densities with a smooth approximation of identity function \( K_{\Delta h} \), \( \left[ \mu(\lambda \mid A ) * K_{\Delta h }  \right]\).

\paragraph{ Kernel Polynomial Method (KPM) }
%
%      1. Chebyshev polynomials
%      2. Moments sampling
%      3. Approximation error estimations
%      4. Computational cost
%      5 . Spikes and filtration /// later  
%

By its very definition, the introduction of the spectral densities \( \mu(\lambda \mid L_k ) \) and \( \left\{ \mu_k(\lambda \mid A ) \right\}\) requires the complete spectral information of the original operator \( L_k \) and, hence, is not immediately computationally more efficient. At the same time, one can use the functional nature of DoS and LDoS to obtain approximation through expansion.

Namely, let \( T_m(x)\) be Chebyhev polynomials of the first kind. Specifically,
\begin{equation}
      T_0(x) = 1, \; T_1(x) = x, \qquad T_{m+1}(x) = 2x T_m(x) - T{m-1} (x)
\end{equation}
which as an orthogonal (with respect to the weight function \(w(x) = \frac{2}{(1+\delta_{0m}) \pi \sqrt{1-x^2}}\)) on \( [-1, 1 ]\). Let us transform the operator \( A \to H  \) such that \( \sigma(H) \in [-1, 1]\) (e.g. \( H = \frac{2}{\lambda_{\max{}}} A - I\)); then one can decompose ˝









\begin{comment}
      \chapter{ Graphs and Simplicial Complexes }

      Simplicial complex \( \mc K \) is a higher-order generalization of the classical graph model with pair-wise interactions with a rich set of topological descriptors. Specifically, let \( \V 0 = \{ v_1, \dots v_{m_0} \} \) be a set of nodes; then \( \mc K \) is a collection of subsets (simplices) \( \sigma \) of nodes from \( \V 0 \) such that all the subsets of \( \sigma \) are simplices in \( \mc K \) too.
      We refer to a simplex made out of \( k + 1 \) nodes \( \sigma = [ v_{i_1}, \ldots v_{i_{k+1}} ] \) as being of order \( k \)%, and write \( \dim \sigma = k \); 
      the set of all the simplices of order \( k \) in the complex \( \mc K \) is denoted by \( \V k \). Thus, \( \V 0 \) are the vertices of \( \mc K \), \( \V 1 \) are edges between pairs of vertices, \( \V 2 \) triangles connecting three vertices, and so on. We let \( m_k = | \V k | \) denote the cardinality of \( \V k\).

      Each set of simplices \( \V k = \left\{ \sigma_1, \dots \sigma_{ m_k } \right\} \) induces a linear space of formal sums over the simplicies \( C_k (\mc K) = \left\{  \sum_{i=1}^{ m_k } \alpha_i \sigma_i  \mid \alpha_i \in \ds R \right\} \) referred to as \textit{chain space}; in particular,  \( C_0 ( \mc K ) \) is known as the space of vertex states and \( C_1 ( \mc K ) \) as the space of edge flows. Simplices of different orders are related through the boundaries operators \( \partial_k \) mapping the simplex to its boundary; formally, \( \partial_k : C_k ( \mc K ) \mapsto C_{k-1} ( \mc K ) \) is defined through the alternating sum:
      \begin{equation*}
            \partial_k [ v_1, v_2, \ldots v_k ] = \sum_{i=1}^k (-1)^{i-1} [ v_1, \ldots v_{i-1}, v_{i+1}, \ldots v_k ] 
      \end{equation*} 
      By fixing an ordering for \( \V k \) we can fix a canonical basis for \( C_k(\mc K)\) and represent each boundary operator as a matrix  \( B_k \in \mathrm{Mat}_{ m_{k-1} \times m_k } \) with exactly \( k \) nonzero entries per each column, being either \( +1 \) or \( -1 \). For these matrices, the fundamental property of topology holds: \textit{the boundary of the boundary is zero}, {\cite[Thm.~5.7]{Lim15}}:
      \begin{equation}
            \label{eq:bkbk1}
            B_k B_{k+1} = 0 
      \end{equation}
      The matrix representation \( B_k \) of the boundary operator \( \partial_k \) requires fixing an ordering of the simplices in \( \V k \) and \( \V{k-1} \). As it will be particularly relevant for the purpose of this work, we emphasize that we order triangles and edges as follows: triangles in \( \V 2 \) are oriented in such a way that the first edge (in terms of the ordering of \( \V 1 \)) in each triangle is positively acted upon by \( B_2 \), i.e.\ the first non-zero entry in each column of \(B_2 \) is \( +1 \), \Cref{fig:orientation}. 

      \begin{figure}[hbtp]
      \centering
      \begin{tikzpicture}
            \begin{scope}[shift={(-0.75, 0)}]
            \draw[fill = liberty, opacity = 0.4] (0,0) -- (1.5,0) -- (0.75, -1.5*3/4) -- cycle; 
            \draw[fill = liberty, opacity = 0.6] (0,0) -- (1.5,0) -- (0.75, 1.5*3/4) -- cycle; 

            \Vertex[x=0, y=0, label = 1, style={color = persimmon}, fontcolor = white, size = 0.4 ]{v1}
            \Vertex[x=1.5, y=0, label = 3, style={color = persimmon}, fontcolor = white, size = 0.4 ]{v2}
            \Vertex[x=0.75, y=-1.5*3/4, label = 2, style={color = persimmon}, fontcolor = white, size = 0.4 ]{v3}
            \Vertex[x=0.75, y=1.5*3/4, label = 4, style={color = persimmon}, fontcolor = white, size = 0.4 ]{v4}
            \Edge[Direct](v1)(v2)
            \Edge[Direct](v1)(v3)
            \Edge[Direct](v1)(v4)
            \Edge[Direct](v3)(v2)
            \Edge[Direct](v2)(v4)
            \node at (0.75, 1.5*3/4*1/3 ) { \AxisRotator[rotate=0] };
            \node at (0.75, -1.5*3/4*1/3 ) { \AxisRotator[rotate=-60] };
            \end{scope}

            \Vertex[x=2.5, y=1.5*3/4, label = 1, style={color = persimmon}, fontcolor = white, size = 0.4 ]{t1}
            \Vertex[x=4, y=1.5*3/4, label = 2, style={color = persimmon}, fontcolor = white, size = 0.4 ]{t2}
            \Vertex[x=2.5, y=0.9*3/4, label = 1, style={color = persimmon}, fontcolor = white, size = 0.4 ]{t3}
            \Vertex[x=4, y=0.9*3/4, label = 3, style={color = persimmon}, fontcolor = white, size = 0.4 ]{t4}
            \Vertex[x=2.5, y=0.3*3/4, label = 1, style={color = persimmon}, fontcolor = white, size = 0.4 ]{t5}
            \Vertex[x=4, y=0.3*3/4, label = 4, style={color = persimmon}, fontcolor = white, size = 0.4 ]{t6}
            \Vertex[x=2.5, y=-0.3*3/4, label = 2, style={color = persimmon}, fontcolor = white, size = 0.4 ]{t7}
            \Vertex[x=4, y=-0.3*3/4, label = 3, style={color = persimmon}, fontcolor = white, size = 0.4 ]{t8}
            \Vertex[x=2.5, y=-0.9*3/4, label = 3, style={color = persimmon}, fontcolor = white, size = 0.4 ]{t9}
            \Vertex[x=4, y=-0.9*3/4, label = 4, style={color = persimmon}, fontcolor = white, size = 0.4 ]{t10}
            \Edge[Direct](t1)(t2)
            \Edge[Direct](t3)(t4)
            \Edge[Direct](t5)(t6)
            \Edge[Direct](t7)(t8)
            \Edge[Direct](t9)(t10)

            \draw[->, line width = 1.0] (2.1, 1.9*3/4)--(2.1, -1.5*3/4);
            \draw[->, line width = 1.0] (2.1, 1.9*3/4)--(4.4, 1.9*3/4);
            \node[ anchor=south ] at ( 3.25, 1.9*3/4 ) { \small orientation };
            \node[ anchor = south, rotate = 90 ] at (2.1, 0.2*3/4 ) { \small ordering };

            \node[] at ( 10.25, 1.0 ) { \( B_2 \textcolor{liberty}{[1, 2, 3 ]} = \overbrace{\textcolor{red}{(+1)} [1, 2]}^{\substack{\text{1st in}\\\text{order}}} + (-1) [1, 3] + (+1) [2, 3] \) };
            \node[] at ( 10.25, -0.4 ) { \( B_2 \textcolor{liberty}{[1, 3, 4 ]} = \underbrace{\textcolor{red}{(+1)} [1, 3]}_{\substack{\text{1st in}\\\text{order}}} + (-1) [1, 4] + (+1) [3, 4] \) };  
      \end{tikzpicture}
      \caption{ Example of the simplicial complex with ordering and orientation: nodes from \( \V 0 \) in orange, triangles from \( \V 2 \) in blue. Orientation of edges and triangles is shown by arrows; the action of \( B_2 \) operator is given for both triangles.\label{fig:orientation}}
\end{figure}


      The following definitions introduce the fundamental concepts of \(k\)-th homology group and \(k\)-th order Laplacian.  See \cite{Lim15} e.g.\ for more details. 

      \begin{definition}[Homology group and higher-order Laplacian]
            Since \( \im B_{k+1} \subset \ker B_k \), the quotient space \( \mc H_k =  \sfrac{ \ker B_k }{ \im B_{k+1}} \), known as \( k\)-th homology group, is correctly defined and the following isomorphisms hold 
      \begin{equation*}
                  \mc H_k \cong \ker B_k \cap \ker B_{k+1}^\top \cong \ker \left( B_k^\top B_k + B_{k+1} B_{k+1}^\top \right).
            \end{equation*}

            The matrix \( L_k = B_k^\top B_k + B_{k+1} B_{k+1}^\top \) is called the \(k\)-th order \emph{Laplacian operator}; the two terms \( \Ld k =  B_k^\top B_k \) and \( \Lu k = B_{k+1} B_{k+1}^\top \) are referred to as the \emph{down-Laplacian} and the \emph{up-Laplacian}, respectively.
      \end{definition}

      The homology group \( \mc H_k \) describes the \(k\)-th topology of the simplicial complex \( \mc K \): \( \beta_k = \dim \mc H_k = \dim \ker L_k \)  coincides exactly with the number of \(k\)-dimensional holes in the complex, known as the \emph{ \(k\)-th Betti number}. In the case \( k = 0 \), the operator \( L_0 = \Lu 0\) is exactly the classical graph Laplacian whose kernel corresponds to the \textit{connected components} of the graph, while \( \Ld 0  = 0 \). For \( k = 1 \) and \( k = 2\), the elements of \( \ker L_1 \) and \( \ker L_2\) describe the simplex 1-dimensional holes and voids respectively, and are frequently used in the analysis of trajectory data,~\cite{schaub2019random,benson2016higher}.


      Although more frequently found in their purely combinatorial form, the definitions of simplicial complexes, homology groups, and higher-order Laplacians admit a generalization to the weighted case. For the sake of generality, in the rest of the work, we use the following notion of weighted boundary operators (and thus weighted simplicial complexes), as considered in e.g.~\cite{guglielmi2023quantifying}.

      \begin{definition}[Weighted and normalised boundary matrices]
            For \textit{weight functions} \( w_k : \V k \mapsto \ds R_+ \cup \{ 0 \} \), define the diagonal weight matrix \( W_k \in \mathrm{Mat}_{ m_k \times m_k } \) as  \( (W_k)_{ii} = \sqrt{w_k(\sigma_i)}\). Then the  weighting scheme for the boundary operators upholding the Hodge algebras~\eqref{eq:bkbk1} is given by:
            \begin{equation}
                  \label{eq:weighting}
                  B_k \mapsto W_{k-1}^{-1} B_k W_k
            \end{equation}
      \end{definition}

      Note that, with the weighting scheme \Cref{eq:weighting}, the dimensionality of the homology group is preserved, \( \dim \ker L_k = \dim \ker \widehat L_k \),~\cite{guglielmi2023quantifying} as well as the fundamental property of topology \Cref{eq:bkbk1}. 

            
\end{comment}
