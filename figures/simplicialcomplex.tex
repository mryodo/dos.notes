\begin{figure}[hbtp]
      \centering
      \begin{tikzpicture}
            \begin{scope}[shift={(-0.75, 0)}]
            \draw[fill = liberty, opacity = 0.4] (0,0) -- (1.5,0) -- (0.75, -1.5*3/4) -- cycle; 
            \draw[fill = liberty, opacity = 0.6] (0,0) -- (1.5,0) -- (0.75, 1.5*3/4) -- cycle; 

            \Vertex[x=0, y=0, label = 1, style={color = persimmon}, fontcolor = white, size = 0.4 ]{v1}
            \Vertex[x=1.5, y=0, label = 3, style={color = persimmon}, fontcolor = white, size = 0.4 ]{v2}
            \Vertex[x=0.75, y=-1.5*3/4, label = 2, style={color = persimmon}, fontcolor = white, size = 0.4 ]{v3}
            \Vertex[x=0.75, y=1.5*3/4, label = 4, style={color = persimmon}, fontcolor = white, size = 0.4 ]{v4}
            \Edge[Direct](v1)(v2)
            \Edge[Direct](v1)(v3)
            \Edge[Direct](v1)(v4)
            \Edge[Direct](v3)(v2)
            \Edge[Direct](v2)(v4)
            \node at (0.75, 1.5*3/4*1/3 ) { \AxisRotator[rotate=0] };
            \node at (0.75, -1.5*3/4*1/3 ) { \AxisRotator[rotate=-60] };
            \end{scope}

            \Vertex[x=2.5, y=1.5*3/4, label = 1, style={color = persimmon}, fontcolor = white, size = 0.4 ]{t1}
            \Vertex[x=4, y=1.5*3/4, label = 2, style={color = persimmon}, fontcolor = white, size = 0.4 ]{t2}
            \Vertex[x=2.5, y=0.9*3/4, label = 1, style={color = persimmon}, fontcolor = white, size = 0.4 ]{t3}
            \Vertex[x=4, y=0.9*3/4, label = 3, style={color = persimmon}, fontcolor = white, size = 0.4 ]{t4}
            \Vertex[x=2.5, y=0.3*3/4, label = 1, style={color = persimmon}, fontcolor = white, size = 0.4 ]{t5}
            \Vertex[x=4, y=0.3*3/4, label = 4, style={color = persimmon}, fontcolor = white, size = 0.4 ]{t6}
            \Vertex[x=2.5, y=-0.3*3/4, label = 2, style={color = persimmon}, fontcolor = white, size = 0.4 ]{t7}
            \Vertex[x=4, y=-0.3*3/4, label = 3, style={color = persimmon}, fontcolor = white, size = 0.4 ]{t8}
            \Vertex[x=2.5, y=-0.9*3/4, label = 3, style={color = persimmon}, fontcolor = white, size = 0.4 ]{t9}
            \Vertex[x=4, y=-0.9*3/4, label = 4, style={color = persimmon}, fontcolor = white, size = 0.4 ]{t10}
            \Edge[Direct](t1)(t2)
            \Edge[Direct](t3)(t4)
            \Edge[Direct](t5)(t6)
            \Edge[Direct](t7)(t8)
            \Edge[Direct](t9)(t10)

            \draw[->, line width = 1.0] (2.1, 1.9*3/4)--(2.1, -1.5*3/4);
            \draw[->, line width = 1.0] (2.1, 1.9*3/4)--(4.4, 1.9*3/4);
            \node[ anchor=south ] at ( 3.25, 1.9*3/4 ) { \small orientation };
            \node[ anchor = south, rotate = 90 ] at (2.1, 0.2*3/4 ) { \small ordering };

            \node[] at ( 10.25, 1.0 ) { \( B_2 \textcolor{liberty}{[1, 2, 3 ]} = \overbrace{\textcolor{red}{(+1)} [1, 2]}^{\substack{\text{1st in}\\\text{order}}} + (-1) [1, 3] + (+1) [2, 3] \) };
            \node[] at ( 10.25, -0.4 ) { \( B_2 \textcolor{liberty}{[1, 3, 4 ]} = \underbrace{\textcolor{red}{(+1)} [1, 3]}_{\substack{\text{1st in}\\\text{order}}} + (-1) [1, 4] + (+1) [3, 4] \) };  
      \end{tikzpicture}
      \caption{ Example of the simplicial complex with ordering and orientation: nodes from \( \V 0 \) in orange, triangles from \( \V 2 \) in blue. Orientation of edges and triangles is shown by arrows; the action of \( B_2 \) operator is given for both triangles.\label{fig:orientation}}
\end{figure}
